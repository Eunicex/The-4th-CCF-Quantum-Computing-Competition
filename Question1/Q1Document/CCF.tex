\documentclass[12pt]{article}
\usepackage{amsmath}    % 数学公式支持
\usepackage{amsfonts}   % 数学符号支持
\usepackage{amssymb}    % 数学符号支持
\usepackage{quantikz}   % 绘制量子电路
\usepackage{ctex}       % 支持中文
\usepackage{caption} % 关键宏包
\captionsetup[figure]{font=small} % 设置caption字体为small

\title{第四届CCF大赛高校赛道 - 决赛}
\date{} % 去掉日期

\begin{document}

\maketitle
\vspace{-2em} % 调整标题位置,减少垂直间距

\section{编程题一}
\subsection{证明量子线路优化方案的正确性}

$H$、$X$、$Y$和$Z$门均为单比特量子门,可通过计算并比较优化前后矩阵表达式是否一致来验证优化的正确性。

\scalebox{1.3}{\textbf{a.}}
\[
    HXH = \frac{1}{\sqrt{2}}\begin{pmatrix} 1 & 1 \\ 1 & -1 \end{pmatrix}
    \begin{pmatrix} 0 & 1 \\ 1 & 0 \end{pmatrix}
    \frac{1}{\sqrt{2}}\begin{pmatrix} 1 & 1 \\ 1 & -1 \end{pmatrix}
    = \begin{pmatrix} 1 & 0 \\ 0 & -1 \end{pmatrix} = Z
\]

\scalebox{1.3}{\textbf{b.}}
\[
    HYH = \frac{1}{\sqrt{2}}\begin{pmatrix} 1 & 1 \\ 1 & -1 \end{pmatrix}
    \begin{pmatrix} 0 & -i \\ i & 0 \end{pmatrix}
    \frac{1}{\sqrt{2}}\begin{pmatrix} 1 & 1 \\ 1 & -1 \end{pmatrix}
    = \begin{pmatrix} 0 & i \\ -i & 0 \end{pmatrix} = -Y
\]

\scalebox{1.3}{\textbf{c.}} 
\[
    HZH = \frac{1}{\sqrt{2}}\begin{pmatrix} 1 & 1 \\ 1 & -1 \end{pmatrix}
    \begin{pmatrix} 1 & 0 \\ 0 & -1 \end{pmatrix}
    \frac{1}{\sqrt{2}}\begin{pmatrix} 1 & 1 \\ 1 & -1 \end{pmatrix}
    = \begin{pmatrix} 0 & 1 \\ 1 & 0 \end{pmatrix} = X
\]

根据以上三个式子,可证得优化前后的线路等价,且都由三个量子门减为单个量子门,优化了量子门个数。

\subsection{计算量子线路酉矩阵}
假设有任意两量子比特量子态$|\psi\rangle$:

\[
|\psi\rangle = |q_1q_0\rangle = \alpha|00\rangle + \beta|01\rangle + \gamma|10\rangle + \delta|11\rangle = \begin{pmatrix} \alpha \\ \beta \\ \gamma \\ \delta \end{pmatrix}, | \alpha\vert ^2+ | \beta\vert ^2+| \gamma\vert^2+| \delta\vert^2=1
\]

\scalebox{1.3}{\textbf{a.}}
\begin{center}
    \scalebox{1.2}{
        \begin{quantikz}
            \lstick{$q_1$} & \ctrl{1} & \qw \\
            \lstick{$q_0$} & \gate{H} & \qw
        \end{quantikz}
    }
\end{center}
    
上述量子电路通过$q_1$控制$q_0$的$H$门操作,当$q_1$为$|0\rangle$时,$q_0$不做任何操作;当$q_1$为$|1\rangle$时,对$q_0$做$H$门操作。
$|\psi\rangle$作用在上述量子电路后得到$|\psi'\rangle$:

\begin{align*}
    |\psi'\rangle &= |q_1'q_0'\rangle 
    = \alpha|00\rangle + \beta|01\rangle + 
    \gamma \left( |1\rangle \otimes \left[ \frac{1}{\sqrt{2}} \left( |0\rangle + |1\rangle \right) \right] \right)
    + \delta \left( |1\rangle \otimes \left[ \frac{1}{\sqrt{2}} \left( |0\rangle - |1\rangle \right) \right] \right) \\    
    &= \alpha|00\rangle + \beta|01\rangle + \frac{1}{\sqrt{2}}(\gamma+\delta)|10\rangle + \frac{1}{\sqrt{2}}(\gamma-\delta)|11\rangle \\
    &= \begin{pmatrix} \alpha \\ \beta \\ \frac{1}{\sqrt{2}}(\gamma+\delta) \\ \frac{1}{\sqrt{2}}(\gamma-\delta) \end{pmatrix}
    = \begin{pmatrix}1 & 0 & 0 & 0 \\ 0 & 1 & 0 & 0 \\ 0 & 0 & \frac{1}{\sqrt{2}} & \frac{1}{\sqrt{2}} \\ 0 & 0 & \frac{1}{\sqrt{2}} & -\frac{1}{\sqrt{2}}\end{pmatrix}
    \begin{pmatrix} \alpha \\ \beta \\ \gamma \\ \delta \end{pmatrix}
\end{align*}

\textbf{因此,对应的酉矩阵为\fbox{$\begin{pmatrix}1 & 0 & 0 & 0 \\ 0 & 1 & 0 & 0 \\ 0 & 0 & \frac{1}{\sqrt{2}} & \frac{1}{\sqrt{2}} \\ 0 & 0 & \frac{1}{\sqrt{2}} & -\frac{1}{\sqrt{2}}\end{pmatrix}$}.}

\vspace{2em}
\scalebox{1.3}{\textbf{b.}}
\begin{center}
    \scalebox{1.2}{
        \begin{quantikz}
            \lstick{$q_1$} & \gate{H} & \qw \\
            \lstick{$q_0$} & \ctrl{-1} & \qw
        \end{quantikz}
    }
\end{center}
 
上述量子电路通过$q_0$控制$q_1$的$H$门操作,当$q_0$为$|0\rangle$时,$q_1$不做任何操作;当$q_0$为$|1\rangle$时,对$q_1$做$H$门操作。
$|\psi\rangle$作用在上述量子电路后得到$|\psi'\rangle$:

\begin{align*}
    |\psi'\rangle &= |q_1'q_0'\rangle 
    = \alpha|00\rangle + \beta \left( \left[ \frac{1}{\sqrt{2}} \left( |0\rangle + |1\rangle \right) \right] \otimes |1\rangle \right) + 
    \gamma|10\rangle + \delta \left( \left[ \frac{1}{\sqrt{2}} \left( |0\rangle - |1\rangle \right) \right] \otimes |1\rangle \right) \\    
    &= \alpha|00\rangle + \frac{1}{\sqrt{2}}(\beta+\delta)|01\rangle + \gamma|10\rangle + \frac{1}{\sqrt{2}}(\beta-\delta)|11\rangle \\
    &= \begin{pmatrix} \alpha \\ \frac{1}{\sqrt{2}}(\beta+\delta) \\ \gamma \\ \frac{1}{\sqrt{2}}(\beta-\delta) \end{pmatrix}
    = \begin{pmatrix}1 & 0 & 0 & 0 \\ 0 & \frac{1}{\sqrt{2}} & 0 & \frac{1}{\sqrt{2}} \\ 0 & 0 & 1 & 0 \\ 0 & \frac{1}{\sqrt{2}} & 0 & -\frac{1}{\sqrt{2}}\end{pmatrix}
    \begin{pmatrix} \alpha \\ \beta \\ \gamma \\ \delta \end{pmatrix}
\end{align*}

\textbf{因此,对应的酉矩阵为\fbox{$\begin{pmatrix}1 & 0 & 0 & 0 \\ 0 & \frac{1}{\sqrt{2}} & 0 & \frac{1}{\sqrt{2}} \\ 0 & 0 & 1 & 0 \\ 0 & \frac{1}{\sqrt{2}} & 0 & -\frac{1}{\sqrt{2}}\end{pmatrix}$}.}

\subsection{$SWAP$门操作}

该部分为提交代码。

\subsection{用$H$门和$CNOT_{1,0}$完成$SWAP$门功能}
由$(3)$可知,两个$CNOT_{1,0}$和一个$CNOT_{0,1}$可以构造$SWAP$门。但由于不可使用$CNOT_{0,1}$门,因此可以通过$H$门和$CNOT_{1,0}$构造一个$CNOT_{0,1}$门,具体线路如下:

\begin{center}
    \begin{minipage}{0.3\textwidth}
    \hspace{2.5em}
    \begin{quantikz}
        \lstick{$q_1$} & \targ{}   & \qw \\
        \lstick{$q_0$} & \ctrl{-1} & \qw
    \end{quantikz}
    \end{minipage}
    $\Longleftrightarrow$
    \hspace{1em}
    \begin{minipage}{0.4\textwidth}
    \begin{quantikz}
        \lstick{$q_1$} & \gate{H} & \ctrl{1} & \gate{H} & \qw \\
        \lstick{$q_0$} & \gate{H} & \targ{}  & \gate{H} & \qw
    \end{quantikz}
    \end{minipage}
\end{center}
    
现证明上述等价替换成立:

设左侧线路对应矩阵$CNOT_{0,1}$,右侧线路对应矩阵$M$,则有:

\begin{align*}
    CNOT_{0,1} = \begin{pmatrix} 1 & 0 & 0 & 0 \\ 0 & 0 & 0 & 1 \\ 0 & 0 & 1 & 0 \\ 0 & 1 & 0 & 0\end{pmatrix}
\end{align*}

\begin{align*}
    M &= (H_1 \otimes H_0) \cdot CNOT_{1,0} \cdot (H_1 \otimes H_0) \\
    &= \frac{1}{2}\begin{pmatrix}1 & 1 & 1 & 1 \\ 1 & -1 & 1 & -1 \\ 1 & 1 & -1 & -1 \\ 1 & -1 & -1 & 1\end{pmatrix}
    \begin{pmatrix}1 & 0 & 0 & 0 \\ 0 & 1 & 0 & 0 \\ 0 & 0 & 0 & 1 \\ 0 & 0 & 1 & 0\end{pmatrix}
    \frac{1}{2}\begin{pmatrix}1 & 1 & 1 & 1 \\ 1 & -1 & 1 & -1 \\ 1 & 1 & -1 & -1 \\ 1 & -1 & -1 & 1\end{pmatrix} \\
    &= \begin{pmatrix} 1 & 0 & 0 & 0 \\ 0 & 0 & 0 & 1 \\ 0 & 0 & 1 & 0 \\ 0 & 1 & 0 & 0\end{pmatrix} = CNOT_{0,1}
\end{align*}

将上述对$CNOT_{0,1}$的等价替换带入$(3)$中的线路,得到的线路如下:

\begin{figure}[h]
    \centering
    \scalebox{1.2}{
    \begin{quantikz}
        \lstick{$q_1$} & \ctrl{1} & \gate{H} & \ctrl{1} & \gate{H} & \ctrl{1} & \qw \\
        \lstick{$q_0$} & \targ{}  & \gate{H} & \targ{}  & \gate{H} & \targ{}  & \qw
    \end{quantikz}
    }
    \caption{$H$门和$CNOT_{1,0}$门构建$SWAP$}
    \label{circuit:swap}
\end{figure}

验证该线路正确性:

设该线路对应的矩阵为$M_{SWAP}$,则有:

\begin{align*}
    M_{SWAP} &= CNOT_{1,0} \cdot M \cdot CNOT_{1,0} \\
    &=  \begin{pmatrix}1 & 0 & 0 & 0 \\ 0 & 1 & 0 & 0 \\ 0 & 0 & 0 & 1 \\ 0 & 0 & 1 & 0\end{pmatrix}
    \begin{pmatrix} 1 & 0 & 0 & 0 \\ 0 & 0 & 0 & 1 \\ 0 & 0 & 1 & 0 \\ 0 & 1 & 0 & 0\end{pmatrix}
    \begin{pmatrix}1 & 0 & 0 & 0 \\ 0 & 1 & 0 & 0 \\ 0 & 0 & 0 & 1 \\ 0 & 0 & 1 & 0\end{pmatrix} \\
    &= \begin{pmatrix} 1 & 0 & 0 & 0 \\ 0 & 0 & 1 & 0 \\ 0 & 1 & 0 & 0 \\ 0 & 0 & 0 & 1\end{pmatrix} = SWAP
\end{align*}

综上,可按\fbox{\textbf{图\ref{circuit:swap}}}方式用$H$门和$CNOT_{1,0}$门构建$SWAP$门。

\end{document}
